% Created 2022-12-31 Sat 20:43
% Intended LaTeX compiler: pdflatex
\documentclass[11pt]{article}
\usepackage[utf8]{inputenc}
\usepackage[T1]{fontenc}
\usepackage{graphicx}
\usepackage{longtable}
\usepackage{wrapfig}
\usepackage{rotating}
\usepackage[normalem]{ulem}
\usepackage{amsmath}
\usepackage{amssymb}
\usepackage{capt-of}
\usepackage{hyperref}
\date{\today}
\title{ml}
\hypersetup{
 pdfauthor={},
 pdftitle={ml},
 pdfkeywords={},
 pdfsubject={},
 pdfcreator={Emacs 28.2 (Org mode 9.5.5)}, 
 pdflang={English}}
\begin{document}

\maketitle
\tableofcontents

\section{Entry test - DSC 305 "Machine Learning"}
\label{sec:orgf74237a}
\subsection{README}
\label{sec:org55ab91f}

We're going to use the statistical programming language R in this
course. Some proficiency with R is assumed though we'll review the
basics at the start.

If you're new to R, I suggest you complete the (free) "\href{https://www.datacamp.com/courses/free-introduction-to-r}{Introduction to
R}" course on the DataCamp platform - this will take about 4 hours of
your time. The course is free - you only have to register with
DataCamp (please use your Lyon email for that).

This entry test covers some of the basics:
\begin{enumerate}
\item R data structures
\item Managing data with R
\item Exploring and understanding data using R
\end{enumerate}

To work through the test, open this \href{https://colab.research.google.com/drive/1FiCejT-5WwsnRcyB7OPfDcP0X-1HwFFi?usp=sharing}{notebook in Google
Colaboratory}\footnote{If you want to start a new R notebook in Google Colaboratory,
you need to enter \texttt{https://colab.to/r}. For an explanation on how to use
Python + R in this environment, \href{https://youtu.be/XasBV68Szk4}{see here}. In the course, we're going
to do literate programming with GNU Emacs + ESS + Org-mode
instead. Colaboratory, RStudio, or DataCamp workspace are all
alternative IDEs. However, none of them give us the freedom and
control that Emacs gives us.} and answer the questions by adding code in the code
blocks and running them one by one as shown \href{https://github.com/birkenkrahe/ml/blob/main/img/colab2.png}{in this figure}.

Alternatively, you can work through \href{https://raw.githubusercontent.com/birkenkrahe/ml/main/org/entrytest.org}{this Org-mode file} using Emacs.

\subsection{R data structures}
\label{sec:org011ffce}

\begin{enumerate}
\item Create a vector named \texttt{subject\_name} to store three patient names,
John Doe, Jane Doe, and Steve Graves.
\begin{verbatim}
...
\end{verbatim}

\item Display the content of \texttt{subject\_name} in five different ways (you
should know at least two ways to do this).
\begin{verbatim}
...
\end{verbatim}

\begin{verbatim}
Error: '...' used in an incorrect context
\end{verbatim}

\item What type of vector is \texttt{subject\_name}? 
\begin{verbatim}
...
\end{verbatim}

\begin{verbatim}
Error: '...' used in an incorrect context
\end{verbatim}
\end{enumerate}
\end{document}