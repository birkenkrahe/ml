% Created 2023-01-03 Tue 10:53
% Intended LaTeX compiler: pdflatex
\documentclass[11pt]{article}
\usepackage[utf8]{inputenc}
\usepackage[T1]{fontenc}
\usepackage{graphicx}
\usepackage{longtable}
\usepackage{wrapfig}
\usepackage{rotating}
\usepackage[normalem]{ulem}
\usepackage{amsmath}
\usepackage{amssymb}
\usepackage{capt-of}
\usepackage{hyperref}
\author{Marcus Birkenkrahe}
\date{\today}
\title{DSC 305.01/CSC 482.01 - Machine Learning - Spring 2023 Syllabus}
\hypersetup{
 pdfauthor={Marcus Birkenkrahe},
 pdftitle={DSC 305.01/CSC 482.01 - Machine Learning - Spring 2023 Syllabus},
 pdfkeywords={},
 pdfsubject={},
 pdfcreator={Emacs 28.2 (Org mode 9.5.5)}, 
 pdflang={English}}
\begin{document}

\maketitle
\section{General Course Information}
\label{sec:org3f7e090}

\begin{itemize}
\item Meeting Times: Tuesday/Thursday, 13:00-13:50 hrs
\item Meeting place: Lyon Building Computer Lab 104
\item Professor: Marcus Birkenkrahe
\item Office: Derby Science Building 210
\item Phone: (870) 307-7254 (Office) / (501 422-4725 (Private)
\item Office Hours: Mon/Wed/Fri 16:15-16:45, Tue/Thu 16:00-16:30
\item Textbook: Machine Learning With R (3e) by Brett Lantz, Packt
Publishing (\href{https://www.packtpub.com/product/machine-learning-with-r-third-edition/9781788295864}{Online: packtpub.com, ebook US\$5})
\end{itemize}

\section{Objectives}
\label{sec:orgadb89f3}

This course is concerned with algorithms that transform information
into actionable intelligence using present-day computers and big
data. We use R, a cross-platform, zero-cost statistical programming
environment that combines a wide range of functions, interfaces to
common machine learning packages, and best-in-class visualization.

\section{Student learning outcomes}
\label{sec:org29694a9}

Students who complete DSC 305, "Machine Learning" (ML), can:

\begin{itemize}
\item Apply un/supervised learning models to big data problems
\item Understand neural networks and Support Vector Machine algorithms
\item Distinguish different methods to make predictions with ML
\item Work with open source libraries like \texttt{Keras} and \texttt{TensorFlow} by Google
\item Master the whole infrastructure for advanced statistical computing
\item Know how to effectively present assignment results
\item Improve data literacy, research and present a project as a team
\end{itemize}

\section{Course requirements}
\label{sec:org3e9437b}

\begin{itemize}
\item Basic proficiency with R is useful (as taught in DSC 105 or obtained
independently on DataCamp "Introduction to R", GitHub's "fasteR", or
Part I of the Book of R by Davies)
\item Imagination, creativity and a visual mind, enjoying finding patterns
and spotting correlations
\item Basic understanding of algorithms and data structures (in any
programming language)
\item Basic understanding of data science infrastructure especially
literate programming methods
\end{itemize}

\section{Grading system}
\label{sec:org9b9b4e9}

\begin{center}
\begin{tabular}{lrrrr}
REQUIREMENT & UNITS & PPU & TOTAL & \% of TOTAL\\
\hline
Final exam & 1 & 100 & 100 & 20.\\
Home assignments & 10 & 10 & 100 & 20.\\
Class assignments & 10 & 10 & 100 & 20.\\
Project sprint reviews & 5 & 20 & 100 & 20.\\
Multiple-choice tests & 10 & 10 & 100 & 20.\\
\hline
TOTAL &  &  & 500 & 100.\\
\hline
\end{tabular}
\end{center}

You should be able to see your current grade at any time using the
Canvas gradebook for the course.

\section{Grading table}
\label{sec:org25534ad}

This table is used to convert completion rates into letter
grades. for the midterm results, letter grades still carry signs,
while for the term results, only straight letters are given (by
rounding up).

\begin{center}
\begin{tabular}{rll}
\hline
\textbf{\%} & \textbf{MIDTERM GRADE} & \textbf{FINAL GRADE}\\
\hline
100-98 & A+ & \\
97-96 & A & A (PASSED -\\
95-90 & A- & VERY GOOD)\\
\hline
89-86 & B+ & \\
85-80 & B & B (PASSED -\\
79-76 & B- & GOOD)\\
\hline
75-70 & C+ & \\
69-66 & C & C (PASSED -\\
65-60 & C- & SATISFACTORY)\\
\hline
59-56 & D+ & \\
55-50 & D & D (PASSED)\\
\hline
49-0 & F & F (FAILED)\\
\hline
\end{tabular}
\end{center}

\section{Standard Policies}
\label{sec:org3b1329f}
\subsection{Honor Code}
\label{sec:orgb33b690}

All graded work in this class is to be pledged in accordance with the
Lyon College Honor Code. The use of a phone for any reason during the
course of an exam is considered an honor code violation.

\subsection{Class Attendance Policy}
\label{sec:org8590db1}

Students are expected to attend all class periods for the courses in
which they are enrolled. They are responsible for conferring with
individual professors regarding any missed assignments. Faculty
members are to notify the Registrar when a student misses the
equivalent of one, two, three, and four weeks of class periods in a
single course. Under this policy, there is no distinction between
“excused” and “unexcused” absences, except that a student may make up
work missed during an excused absence. A reminder of the college’s
attendance policy will be issued to the student at one week, a second
reminder at two weeks, a warning at three weeks, and notification of
administrative withdrawal and the assigning of an “F” grade at four
weeks. Students who are administratively withdrawn from more than one
course will be placed on probation or suspended.

\subsection{Academic Support}
\label{sec:org7051595}

The Morrow Academic Center (MAC) helps students who want to improve
grades by providing peer-led services including Supplemental
Instruction (SI), tutoring, the Writing Center, and academic coaching
as well providing 24-hour, online tutoring for all subjects through
Tutor.com. A schedule of peer-led services is available at
lyon.edu/mac and Tutor.com is accessed through courses in
Schoology. Contact Donald Taylor, Director of Academic Support, at
870-307-7319 or donald.taylor@lyon.edu for more information about MAC
services.

\subsection{Technology Support}
\label{sec:org8161210}

For general technology support, you can contact the IT department by
emailing support@lyon.edu or by calling 870-307-7555. For assistance
with classroom-related technologies, such as the learning management
system (LMS), you can request support using the methods above, or you
can contact sarah.williams@lyon.edu directly for assistance. Your
course content will be accessible digitally using either the Schoology
or Canvas LMS. Both LMS platforms will use your myLyon credentials for
your student login.

\begin{itemize}
\item For Canvas, login at lyon.instructure.com
\item For Schoology, login at lyon.schoology.com
\end{itemize}

\subsection{Disabilities}
\label{sec:orga69f281}

Students seeking reasonable accommodations based on documented
learning disabilities must contact Interim Director of Academic
Support Donald Taylor in the Morrow Academic Center at (870) 307-7019
or at donald.taylor@lyon.edu.

\subsection{Harassment, Discrimination, and Sexual Misconduct}
\label{sec:org3f2bec7}

Lyon College seeks to provide all members of the community with a safe
and secure learning and work environment that is free of crime and/or
policy violations motivated by discrimination, sexual and bias-related
harassment, and other violations of rights. The College has a
zero-tolerance policy against gender-based misconduct, sexual assault,
and interpersonal violence toward any member or guest of the Lyon
College community. Any individual who has been the victim of an act of
violence or intimidation is urged to make an official report by
contacting a campus Title IX coordinator or by visiting
www.lyon.edu/file-a-title-ix-report. A report of an act of violence or
intimidation will be dealt with promptly. Confidentiality will be
maintained to the greatest extent possible within the constraints of
the law. For more information regarding the College’s Title IX
policies and procedures, visit www.lyon.edu/title-ix.

\subsection{Mental \& Behavioral Health}
\label{sec:org8b5b518}

Lyon College is dedicated to ensuring each student has access to
mental and behavioral health resources. The College’s Mental and
Behavioral Health Office is located in Edwards Commons and is
partnered with White River Health System’s Behavioral Health
Clinic. The office is committed to helping the Lyon community achieve
maximum mental and behavioral wellness through both preventative and
reactive care. A full-time, licensed, professional counselor provides
counseling, consultations, outreach, workshops, and many more mental
and behavioral services to Lyon students, faculty, and staff at no
cost. The Mental and Behavioral Health Office also provides access to
White River Health System’s services and facilities, including
medication management and in-patient and out-patient care. To make an
appointment, contact counseling@lyon.edu.

\subsection{College-Wide COVID-19 Policies}
\label{sec:orgd466e8a}

The College does not require masks in instructional and meeting spaces
inside academic buildings. However, if instructors require masks in
their classroom, lab, or studio, then students and guests must comply
with that requirement.  Vaccines are strongly encouraged for all
faculty, staff, and students. Vaccines are not mandated for Lyon
College community members, although there may be specific courses
involving interactions with vulnerable, external populations where a
vaccine may be required.  The College will continue to offer
symptomatic testing for students, faculty and staff.

\subsection{Details}
\label{sec:org2dafc45}

Details specific to this course may be found in the subsequent pages
of this syllabus. Those details will include at least the following:
\begin{itemize}
\item A description of the course consistent with the Lyon College
catalog.
\item A list of student learning outcomes for the course.
\item A summary of all course requirements.
\item An explanation of the grading system to be used in the course.
\item Any course-specific attendance policies that go beyond the College
policy.
\item Details about what constitutes acceptable and unacceptable student
collaboration on graded work.
\item A clear statement about which LMS is being used for the course.
\end{itemize}

\subsection{Learning Management System (LMS)}
\label{sec:org5e2b12e}

We will use Canvas in this course.
\section{Assignments and Honor Code}
\label{sec:orgc8b7c31}

There will be several assignments during the summer school,
including programming assignments and multiple-choice tests. They
are due at the beginning of the class period on the due date. Once
class begins, the assigment will be considered one day late if it
has not been turned in.  Late programs will not be accepted without
an extension. Extensions will \textbf{not} be granted for reasons such as:

\begin{itemize}
\item You could not get to a computer
\item You could not get a computer to do what you wanted it to do
\item The network was down
\item The printer was out of paper or toner
\item You erased your files, lost your homework, or misplaced your
flash drive
\item You had other coursework or family commitments that interfered
with your work in this course
\end{itemize}

Put “Pledged” and a note of any collaboration in the comments of any
program you turn in. Programming assignments are individual efforts,
but you may seek assistance from another student or the course
instructor.  You may not copy someone else’s solution. If you are
having trouble finishing an assignment, it is far better to do your
own work and receive a low score than to go through an honor trial and
suffer the penalties that may be involved.

What is cheating on an assignment? Here are a few examples:

\begin{itemize}
\item Having someone else write your assignment, in whole or in part
\item Copying an assignment someone else wrote, in whole or in part
\item Collaborating with someone else to the extent that your
submissions are identifiably very similar, in whole or in part
\item Turning in a submission with the wrong name on it
\end{itemize}

What is not cheating?  Here are some examples:
\begin{itemize}
\item Talking to someone in general terms about concepts involved in an
assignment
\item Asking someone for help with a specific error message or bug in
your program
\item Getting help with the specifics of language syntax or citation
style
\item Utilizing information given to you by the instructor
\end{itemize}

Any assistance must be clearly explained in the comments at the
beginning of your submission.  If you have any questions about this,
please ask or review the policies relating to the Honor Code.

Absences on Days of Exams: Test “make-ups” will only be allowed if
arrangements have been made prior to the scheduled time.  If you are
sick the day of the test, please e-mail me or leave a message on my
phone before the scheduled time, and we can make arrangements when
you return.

\section{Attendance policy}
\label{sec:org55c94c5}

In accordance with college policy, if you miss 4 weeks of class, you
fail the course automatically. Any missed meetings result in an \href{https://catalog.lyon.edu/class-attendance}{"Early
Alert" report}.

You should take care not to miss consecutive sessions if at all
possible - otherwise you risk losing touch with the class and falling
behind.
\section{Important Dates}
\label{sec:orgbd582d3}

\begin{center}
\begin{tabular}{lll}
DATE & DAY & DESCRIPTION\\
\hline
3 January & Tuesday & Last day to deposit for '22 spring semester\\
10 January & Tuesday & Classes begin\\
16 January & Monday & MLK Day - no classes\\
17 January & Tuesday & Last day to add a class\\
24 January & Tuesday & Last day to drop without record of course\\
 &  & Last day to declare a course pass-fail\\
 &  & Deadline for removal of incompletes\\
25-28 February & Saturday-Tuesday & Mental-Health break (no classes)\\
1 March & Wednesday & Mid-term grades available by noon\\
8 March & Wednesday & Lst day to drop a course with a "W"\\
18-26 March & Saturday-Sunday & Spring break\\
7-9 April & Friday-Sunday & Easter break\\
18 April & Tuesday & Honors Convocation\\
4 May & Wednesday & Last day of spring classes\\
4-7 May & Thursday-Sunday & Final exams for graduating seniors\\
 &  & (start 6pm Thu, no exams before 1pm Sun)\\
5-10 May & Thursday-Tuesday & Final exams for non-graduating students\\
 &  & (no exams before 1pm on Sunday)\\
9 May & Tuesday & Senior grades due by noon\\
12 May & Friday & Baccalaureate\\
13 May & Saturday & Spring commencement\\
17 May & Wednesday & All final grades due by noon\\
\end{tabular}
\end{center}

\section{Schedule and session content}
\label{sec:org0e9b9fe}

Lectures and lab sessions are aligned with the content of the 10
DataCamp lessons that need to be completed in the course of the term.

\begin{center}
\begin{tabular}{rlll}
WEEK & DATE & DATACAMP ASSIGNMENT & TESTS\\
\hline
1 & Jan 10,12 &  & \\
\hline
2 & Jan 17,19 & What is Machine Learning? & Test 1\\
\hline
3 & Jan 24,26 & Machine Learning Models & Test 2\\
\hline
4 & Jan 31, Feb 2 & k-Nearest Neighbors (kNN) & Test 3\\
\hline
5 & Feb 7,9 & \textbf{Sprint review 1: literature review} & \\
\hline
6 & Feb 14,16 & Naive Bayes & Test 4\\
\hline
7 & Feb 21,23 & Logistic Regression & Test 5\\
\hline
8 & Mar 2 & Classification Trees & Test 6\\
\hline
9 & Mar 7,9 & \textbf{Sprint review 2: methodology} & \\
\hline
10 & Mar 14,16 & Unsupervised learning: clustering & Test 7\\
\hline
11 & Mar 28,30 & Hierarchical clustering & Test 8\\
\hline
12 & Apr 4,6 & Dimensionality reduction & Test 9\\
\hline
13 & Apr 11,13 & \textbf{Sprint review 3: abstract} & \\
\hline
14 & Apr 18,20 & Unsupervised learning case study & Test 10\\
\hline
15 & Apr 25,27 & \textbf{Sprint review 4: final presentation} & \\
\hline
16 & May 2 &  & \\
\hline
\end{tabular}
\end{center}
\end{document}